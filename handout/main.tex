% LTeX: language=de-DE
\documentclass[ngerman]{scrreport}
\usepackage{babel}
\usepackage{csquotes}
\usepackage{hyperref}

\usepackage{graphicx}
\usepackage{xcolor}
\usepackage[
backend=biber,
style=alphabetic,
sorting=ynt
]{biblatex}
\addbibresource{beispiel.bib}
\usepackage[de-DE, showdow]{datetime2} % make handling dates easier
\date{\DTMDate{2023-06-23}}


\title{\LaTeX{}-Handout}
\author{Juliana Kunz \and Benedikt Peetz \and Jonathan Steiner}
\date{\DTMToday{}}

\begin{document}

\maketitle
\tableofcontents
\vspace*{\fill}
Copyright \textcopyright{} Juliana Kunz, Benedikt Peetz und Jonathan Steiner 2023\\
\ \\
Dieses Werk ist lizenziert unter den Bedingungen der CC BY-SA 4.0.
Der Lizenztext ist online unter \url{http://creativecommons.org/licenses/by-sa/4.0/legalcode} abrufbar.

\chapter{\centering\LaTeX: Einführung}

\section{Grundprinzip}
Um mit \LaTeX ein Dokument zu erstellen, schreibt man nach eine unformatierte \texttt{.tex} Datei nach \LaTeX-Syntax, die danach von einer \LaTeX-Installation zum fertigen stilisierten Dokument kompiliert wird.
Besonders für Dokumente, in denen sich Fließtext und komplizierte Grafiken abwechseln, ist \LaTeX hilfreich. Mit erweiternden Paketen können u.~A. komplizierte mathematische und chemische Formeln, Schachbretter, musikalische Noten, etc. dargestellt werden.
Dabei können mit \LaTeX viele verschiedene Dokumenttypen erstellt werden, u.~a. Bücher, wissenschaftliche Artikel, Präsentationen, \dots

\section{Syntax}

Generell folgen \LaTeX-Befehle folgendem Syntax:

\color{blue}
\begin{quote}
	\verb|\befehl[optionale Param.]{verpflichtende Param.}|{\color{black}, \\z.~B.} \verb|\textit{kursiver Text}|
\end{quote}
\color{black}
Nicht alle Befehle haben aber Parameter.

Umgebungen, z.~B. Listen, werden folgendermaßen erstellt:

\color{blue}
\begin{quote}
	\verb|\begin{typ}| \dots \verb|\end{typ}|{\color{black}, \\z.~B.} \verb|\begin{enumerate}| \dots \verb|\end{enumerate}|
\end{quote}
\color{black}

\section{Aufbau}

\LaTeX-Dateien sind in zwei Bereiche unterteilt: die Präambel und der Inhaltsteil. In ersterer werden die Art des Dokuments ({\color{blue}\verb|\documentclass{}|}), verschiedene Metadaten (u.~A. {\color{blue}\verb|\author|}), und erste Layout-Informationen festgelegt.
Auch eigene Befehle können mit {\color{blue}\verb|\newocmmand{}{}|} erstellt werden.

Im zweiten Abschnitt wird dann der eigentliche Inhalt erstellt. Dabei können die Informationen wie Titel, Autor etc. mit {\color{blue}\verb|\maketitle|} dargestellt werden. Mit Kapiteln, Bereichen etc. kann das Dokument gegliedert werden, sowie mit {\color{blue}\verb|\tableofcontents|} ein Inhaltsverzeichnis generiert werden.
Auch Quellenangaben und generieren einer Bibliografie ist möglich. Zudem können externe \texttt{.tex}-Dateien mit {\color{blue}\verb|\imput{}|} ins Dokument eingebunden werden.

Im Folgenden werden nun die wichtigsten Befehle für den Inhaltsbereich erklärt.

\chapter{\centering Nützliche \LaTeX{}-Befehle}

\section{Schriftschnitte}
Um einzelne Worte \textit{kursiv} zu schreiben, nutzt man den Befehl {\color{blue}\verb|\textit{...}|}.\\
\textbf{Fett drucken:} {\color{blue}\verb|\textbf{...}|}\\
\underline{Unterstreichen:} {\color{blue}\verb|\underline{...}|}\\
Man kann auch unabhängig vom restlichen Text einzelne Wörter \emph{hervorheben} durch {\color{blue}\verb|\emph{...}|}. Dabei wird das Wort meist kursiv geschrieben, außer der ganze Satz wurde schon in kursiv geschrieben. Es passt sich auch anderen Schriftschnitten an.


\section{Position der Elemente}
Durch den Befehl {\color{blue}\verb|\centering|} kann man Texte und Bilder zentrieren, unter anderem muss man manchmal Umgebungen erstellen, also {\color{blue}\verb|\begin{center}|} und {\color{blue}\verb|\end{center}|} notieren.\\
\raggedright Der Befehl {\color{blue}\verb|\raggedright|} lässt den gesamten folgenden Text linksbündig schreiben.\\
\raggedleft Der Befehl {\color{blue}\verb|\raggedleft|} lässt den gesamten folgenden Text rechtsbündig schreiben.
\raggedright

\section{Paragrafen}
Um einen neuen Paragrafen zu erstellen, kann man entweder den Befehl {\color{blue}\verb|\newline|} \newline verwenden.\\ Oder {\color{blue}\verb|\\|}, dann wird auch \\ in der nächsten Zeile weitergeschrieben.

\section{Kapitel und Abschnitte}
Kapitel werden durch den Befehl {\color{blue}\verb|\chapter{}|} gebildet, dabei wird auch jedes Mal dem Namen eine Zahl zugeordnet, möchte man dies umgehen, schreibt man {\color{blue}\verb|\chapter*{}|}.\\
Das Gleiche gilt auch für die Abschnitte {\color{blue}\verb|\section{}|} und {\color{blue}\verb|\subsection{}|}, der Name ist dabei fett gedruckt und größer als der normale Text.

\section{Kommentare}
Kommentare werden mit einem {\color{blue}\ \%} - Zeichen eingeleitet und dienen dem Schreiber.
\newline
\underline{Beispiel:} {\color{blue}\verb|% Dieser Befehl dient dem Einfügen von Bildern.|}

\section{Kopfzeilen und Fußnoten}
Für die Kopfzeile und Fußnote muss ein Paket durch {\color{blue}\verb|\usepackage{fancyhdr}|} eingefügt werden, sowie das Ändern der Seite durch {\color{blue}\verb|\pagestyle{fancy}|}.\\
Dabei fügen die Befehle:\\
{\color{blue}\verb|\fancyhead[]{} \fancyfoot[]{}|} Kopfzeilen und Fußnoten hinzu.\\
In die eckigen Klammern kommt dabei erstmal die Position der Erweiterung durch {\color{blue}L R C} und dazu kommt dann ein {\color{blue}O} für eine ungerade Seite oder {\color{blue}E} für eine gerade Seite.
Diese Erweiterungen tauchen auf jeder ungerade oder gerade Seiten auf, das Prinzip ist bei der Dokumentenklasse \texttt{book} enthalten, man kann es durch {\color{blue}\verb|[twoside]{report}|} bei \texttt{reports} oder \texttt{articles} ergänzen, mit Kapiteln sind die Ergänzungen  nicht verträglich.\\
Schlussendlich sieht ein solcher Befehl so aus:\\
{\color{blue}\verb|\fancyfoot[LO, RE]{Diese Fußnote taucht linksbündig auf allen ungeraden,|}\\
{\color{blue}\verb|rechtsbündig auf allen geraden Seiten auf}|}.

\section{Schriftgrößen}
Um Worte oder einzelne Zeichen zu vergrößern oder verkleinern, sind die Befehle {\color{blue}\verb|{\huge} {\large} {\small} {\footnotesize}|} von Nutzen.\\
Wenn man ganze Texte vergrößer oder verkleinern will, lässt man die geschwungenen Klammern weg und schreibt am Ende {\color{blue}\verb|\normalsize|}.

\section{Grafiken}
Bilder o.~Ä. können erst eingefügt werden, wenn ein Paket mit vordefinierten Befehlen am Anfang mit eingebunden wird, hier wäre es {\color{blue}\verb|\usepackage{graphicx}|}.
\\Um das Bild einfügen zu können, muss man angeben, wo die Grafik in deinen Dateien liegt.\\
\underline{Den Pfad zum Bild angeben:} {\color{blue}\verb|\graphicspath{Downloads/beispiel}|}.\\
Das Bild \texttt{beispiel} liegt hier in einem Ordner \texttt{Downloads}.\\ Zusätzlich kann die Größe des Bildes durch {\color{blue} \verb|"width="|} angegeben werden, in diesem Fall ist das Bild 50 \% so groß, wie der Text, auch kann man sich auf die Texthöhe mit {\color{blue}\verb|"height=\textheight"|} beziehen. \\
\underline{Einfügen des Bildes:} {\color{blue}\verb|\includegraphics[width=0.5\textwidth]{LaTeX-logo}|}\\
\underline{Beispiel:}
\graphicspath{{resources}}
\begin{figure}[h]
	\includegraphics[width=0.5\textwidth]{LaTeX-logo}
	\centering
\end{figure}\\
Mit {\color{blue}\verb|\begin{figure}[h]|} und {\color{blue}\verb|\end{figure}|} kann man noch den Ankerpunkt des Bildes angeben.\\
\begin{description}
	\item [{\color{blue}t}] am Anfang der Seite
	\item [{\color{blue}b}] am Ende der Seite
	\item [{\color{blue}h}] ungefähr an der Stelle, an welcher man es im Code geschrieben hat\\
\end{description}
\vspace{0.5cm}
Das Bild kann auch benannt werden, einer Zahl zugewiesen werden und auf die Abbildung kann verwiesen werden, der Label- und Captionbefehl muss dabei im {\color{blue}\verb|\begin{figure}|} Teil stehen.\\
\graphicspath{{resources/}}
\begin{figure}[h]
	\centering
	\includegraphics[width=0.75\textwidth]{LaTeX-logo}
	\begin{center}Abbildung 1: Mit {\color{blue}\verb|\caption{}|} erstellt man diesen Text.\end{center}
\end{figure}
\raggedright
\label{fig:LaTeX-logo}
Mit {\color{blue}\verb|\label{fig:LaTeX-logo}|} wurde dem Bild eine Nummer zugewiesen und mit {\color{blue}\verb|\ref{fig:LaTeX-logo.png}|} wird diese Zahl hier ausgegeben: 1.\\

\section{Dateien einfügen}
Ähnlich wie Grafiken können auch andere \LaTeX{}-Dateien eingefügt werden.
Dabei wird der Befehl {\color{blue}\verb|\input{}|} genutzt, in die Klammer kommt der Name der anderen Datei mit \texttt{.tex}, diese darf keine {\color{blue}\texttt{documentclass}} beinhalten oder {\color{blue}\verb|\begin{document}|}, sowie {\color{blue}\verb|\end{document}|}.
Gestaltet sie so, dass sie ein direkter Ausschnitt aus der Datei sein könnte.
Wenn sich die Dateien nicht im gleichen Ordner befinden, muss der Ordnername durch {\color{blue}\verb|{Ordner/Dateiname.tex}|} angegeben werden.

\section{Listen}
Bei Listen muss zuerst eine Umgebung erstellt werden, also ein Bereich, in welchem die Liste sich befindet.
\begin{itemize}
	\item Zuerst schreibt man also: {\color{blue}\verb|\begin{itemize}|} oder {\color{blue}\verb|{enumerate}|}.
	\item Mit {\color{blue}\verb|\item|} werden dann die einzelnen Listenpunkte definiert.
	\item Um die Liste zu beenden, schreibt man dann: {\color{blue}\verb|\end{itemize}|} oder {\color{blue}\verb|{enumerate}|}.
\end{itemize}
Man kann
\begin{enumerate}
	\item mit {\color{blue}\texttt{itemize}} ungeordnete Listen (mit Aufzählungspunkt) und
	\item mit {\color{blue}\texttt{enumerate}} geordnete Listen
\end{enumerate}
darstellen.

\section{Mathematik}
Um auch etwas Mathematik hier einfließen zu lassen, kann man mathematische Formeln darstellen. Wenn man das im Satz tun möchte, passiert das am leichtesten durch das Einklammern mit den {\color{blue}\verb|$…$|} - Zeichen.\\
Möchte man eine mathematische Formel als einzigen Inhalt einer Zeile haben, nutzt man {\color{blue}\verb|\[...\]|} zum Einklammern.\\
\underline{Beispiel:}\\
Der Flächeninhalt eines Kreises wird durch die Formel $A=\pi\cdot r^2$ beschrieben.\\
Der Flächeninhalt eines Kreises wird durch die Formel:
\[A=\pi\cdot r^2\] beschrieben.

\section{Tabellen}
Um eine Tabelle zu erstellen, braucht es wieder die Eingrenzung einer Umgebung, durch {\color{blue}\verb|\begin{tabular}{c c c}|} oder {\color{blue}\verb|{c r c}|} oder {\color{blue}\verb|{l r l}|}, usw.\\
Die zweite Klammer gibt an, ob der Inhalt einer Tabellenzelle {\color{blue}\underline{c}}entriert, auf der {\color{blue}\underline{l}}inken oder auf der {\color{blue}\underline{r}}echten Seite ist.\\
In diesem Fall gibt es drei×drei Tabellenzellen.\\
Dann werden die einzelnen Tabellen beschrieben durch:\\
\color{blue}
\begin{verbatim}
cell1 & cell2 & cell3\\
cell4 & cell5 & cell6\\
cell7 & cell8 & cell9\\
\end{verbatim}
\color{black}
Durch {\color{blue}\verb|\end{tabular}|} wird die Tabellenumgebung geschlossen.
\vspace{4cm}

\subsection{Grenzen}
Um die Tabelle vertikal einzugrenzen kann man am Anfang: {\color{blue}\begin{verbatim}{c|c|c|}\end{verbatim}} notieren und um horizontal einzugrenzen, schreibt man am Anfang und Ende der Zellenbeschreibung {\color{blue}\verb|\hline|}, das kann man auch doppelt schreiben.\\
\underline{Beispiel:}\\
\vspace{0.5cm}
\begin{tabular}{||c|c|c||}
	\hline\hline
	cell1 & cell2 & cell3 \\
	\hline
	cell4 & cell5 & cell6 \\
	\hline
	cell7 & cell8 & cell9 \\
	\hline\hline
\end{tabular}
\color{blue}
\begin{verbatim}
\begin{tabular}{||c|c|c||}
\hline\hline
cell1 & cell2 & cell3\\
\hline
cell4 & cell5 & cell6\\
\hline
cell7 & cell8 & cell9\\
\hline\hline
\end{tabular}\\
\end{verbatim}
\color{black}

\subsubsection{Zellenverschmelzung}
Zellenfelder kann man durch den Befehl: {\color{blue}\verb|\multicolumn{2}}{c}{cell 2}|} miteinandervereinen.\\
\underline{Beispiel:}\\
\vspace{0.5cm}
\begin{tabular}{c|c|c}
	\hline
	\multicolumn{3}{c}{cell1,2,3} \\
	\hline
	cell4 & cell5 & cell6         \\
	cell7 & cell8 & cell9         \\
\end{tabular}\\
\vspace{0.5cm}

\section{Interessante Befehle}

\subsection{Abstände}
Man kann sowohl horizontale als auch vertikale Abstände einfügen, das macht man mit den Befehlen: {\color{blue}\verb|\hspace{1cm}|} oder {\color{blue}\verb|\vspace{1cm}|}.\\
Das kann dann so \hspace{3cm} aussehen oder so \vspace{1cm}\\ aussehen.

\subsection{Übergreifende Schriftart}
Durch {\color{blue}\verb|\textit{}|} können auch ganze Sätze kursiv geschrieben werden.
Für kursiv benutzt man {\color{blue}\verb|textit|}, für fett {\color{blue}\verb|\textbf|} und bei unterstrichenen Sätzen {\color{blue}\verb|\underline|}, dabei kann ein ganzer Satz von den Klammern umschlossen werden. \\
\underline{Beispiel:}\\
\textit{Dieser ganze Satz ist kursiv!!}\\
\textbf{Dieser ganze Satz ist fett gedruckt!!}

\subsection{Befehle als normalen Text}
Um etwa diesen Text in \LaTeX{} erstellen und dabei die Befehle nennen zu können, muss man wissen, wie man Befehle notiert, ohne dass diese als Befehl ausgeführt werden.\\
Jenes tut man durch den Befehl:\\
\color{blue}
\begin{verbatim} \verb|…| \end{verbatim}
\color{black}
Falls man ganze Bereiche von Befehlen als Text möchte, erstellt man eine Umgebung mit {\color{blue}\verb|\begin{verbatim}|} und beendet sie mit {\color{blue}\verb|\end{verbatim}|}.

\subsection{Farbe}
In diesem Handout nutzte ich die farbliche Hervorhebung von Befehlen, davor muss man das Paket {\color{blue}\texttt{xcolor}} durch {\color{blue}\verb|\usepackage{xcolor}|} einfügen.\\
Durch {\color{blue}\verb|\color|} wird dann der ganze folgende Text in der Wahlfarbe eingefärbt.\\
Auch hier kann man alternativ eine Umgebung mit {\color{blue}\verb|\begin{red}|} und mit {\color{blue}\verb|\end{red}|} erstellen.\\
Einzelne Worte können mit {\color{blue}\verb|{\color{red} ...}|} koloriert werden, um einen farblichen Kasten zu erstellen, benutzt man den Befehl {\color{blue}\verb|\colorbox{red}{}|}.\\
\underline{Beispiele:}\\
\color{blue} Dieser ganze Satz ist blau.\\
\color{black}
In diesem Satz ist nur das Wort {\color{red} Apfel} rot.\\
Und hier ist das letzte Wort in einem farblichen \colorbox{orange}{Kasten}.\\

\section{Quellenangaben}
Für das Aufzeigen der Quellen wird einer Paket gebraucht, es lautet wie folgt:\\
{\color{blue}
\begin{verbatim}
\usepackage[
backend=biber,
style=alphabetic,
sorting=ynt
]{biblatex}
\addbibresource{beispiel.bib}
\end{verbatim}}
Dabei sorgt {\color{blue} \texttt{backend=biber}} für die Sortierung der Auflistung,\\
{\color{blue} \texttt{style=standard}} definiert den Stil und {\color{blue} \texttt{sorting=ynt}} legt den Sortierstil fest, hier ist die Abstufung {\color{blue}\underline{y}}ear {\color{blue}\underline{n}}ame {\color{blue}\underline{t}}itel.
Die Bibliothek wird dann als neue Datei mit dem Namen {\color{blue} \texttt{beispiel.bib}} erstellt und hinzugefügt durch {\color{blue}\verb|\addbibresource{beispiel.bib}|}.\\
Um dann Quellen einzufügen, muss man sie in einer separaten Datei erstellen und dann per Befehl einfügen.\\
\underline{Beispiel:}\\
\color{blue}
\begin{verbatim}
@book{harrypotter,
    author    = "Joanne Kathleen Rowling",
    title     = "Harry Potter und der Stein der Weisen,
    year      = "2005",
    publisher = "Carlsen",
    isbn   = "978-3-551-35401-3",
    keywords  = "harrypotter"
}
\end{verbatim}
\color{black}
Zuerst wird durch {\color{blue} \texttt{@}} die Art der Quelle angegeben (hier: {\color{blue} \texttt{book}}), dann ein Quellreferenzname (hier: {\color{blue} \texttt{harrypotter}}), welcher beim Quellverweis benutzt wird, dann in diesem Fall der Autor; der Titel; das Jahr; … am Ende wird noch ein Keyword angegeben, damit man Quellen einer Art je nach Keyword oder Quellenart sortieren kann.\\
Ohne {\color{blue}\verb|\cite{harrypotter}|} kann keine Quelle eingefügt werden, der Quellverweis sieht wie folgt aus.\\

\begin{quote}
	Das Lieblingsbuch der 2000er war \textit{Harry Potter und der Stein der Weisen} \autocite{harrypotter}, geschrieben von J.K. Rowling, diese ist heute für ihre transfeindlichen Aussagen und Aktionen bekannt.
\end{quote}

Das Einfügen durch {\color{blue}\verb|\printbibliography[title={Bücher]|} sieht dann folgendermaßen aus (siehe letzte Seite).\\

Je nach Art der Quelle kann auch eine URL hinzugefügt werden, usw. \\
Um Quellen zu sortieren, kann man bei {\color{blue} \verb|\printbibliography|} nach z.~B. den angegebenen Quelltypen sortieren, mit {\color{blue}\verb|\printbibliography[type=book,title={nur Buchreferenzen}]|}\\
kann man auch nach den {\color{blue} \texttt{keywords}} sortieren, falls zum Beispiel mehrere Harry Potter Teile in einem Artikel erwähnt werden, notiert man bei allen als Schlüsselwort harrypotter und sortiert diese dann so: {\color{blue}\verb|\printbibliography[keyword={harrypotter},title={Harry Potter Serie}]|}\\
\vspace{4cm}

\section{Für deutschsprachige Dokumente}
Um einige Elemente der \LaTeX-Formatierung an die deutsche Sprache anzupassen, müssen im Präambel einige Änderungen vorgenommen werden.

\color{blue}
\begin{verbatim}
    \documentclass[ngerman]{article}

    \usepackage[T1]{fontenc} % Nur wenn pdflatex genutzt wird
    \usepackage{babel}
    \usepackage{csquotes}
\end{verbatim}
\color{black}

Das Paket {\color{blue}\texttt{babel}} übersetzt generierte Texte wie {\color{blue}\textit{table of contents}} ins Deutsche, das Paket {\color{blue}\texttt{csquotes}} sorgt für deutsche Anführungszeichen. Damit diese Pakete wissen, dass das Dokument deutsch ist, wird {\color{blue}\texttt{ngerman}} (d.~h. \underline{n}eue \underline{deutsche} Rechtschreibung) als Parameter in {\color{blue}\verb|\documentclass[]{}|} übergeben.

\section{Für weitere Informationen}
Mehr Informationen zu den einzelnen Befehlen sowie weitere Befehle findet ihr auch auf Englisch unter \url{https://www.overleaf.com/learn} oder auf Deutsch unter \url{https://www.learnlatex.org/de/}.\\
Falls es spezifische Probleme gibt, findet man die Antwort meist auf \url{https://tex.stackexchange.com}.

\subsection{Quellenverweis}
LaTeX-logo : \url{https://de.m.wikipedia.org/wiki/Datei:LaTeX_logo.svg}
\vspace{2cm}

{\huge Viel Spaß mit \LaTeX{!!!}}

\printbibliography[title={Bücher}]

\end{document}
