% LTeX: language=de-DE
\documentclass{report}
\usepackage{graphicx}
\usepackage{xcolor}

\title{\LaTeX{}-Handout}
\author{Juliana Kunz \and Benedikt Peetz \and Jonathan Steiner}
\date{}

\begin{document}
\maketitle

\chapter*{\centering Zusätzliche \LaTeX{}-Befehle}
\section*{Position der Elemente}
Durch den Befehl {\color{blue}\verb|\centering|} kann man Texte und Bilder zentrieren, unter anderem muss man manchmal Umgebungen erstellen, also {\color{blue}\verb|\begin{center}|} und {\color{blue}\verb|\end{center}|} notieren.\\
\raggedright Der Befehl {\color{blue}\verb|\raggedright|} lässt den gesamten folgenden Text linksbündig schreiben.\\
\raggedleft Der Befehl {\color{blue}\verb|\raggedleft|} lässt den gesamten folgenden Text rechtsbündig schreiben.

\raggedright

\section*{Paragraphen}
Um einen neuen Paragraphen zu erstellen, kann man entweder den Befehl {\color{blue}\verb|\newline|} \newline verwenden.\\ Oder {\color{blue}\verb|\\|}, dann wird auch \\ in der nächsten Zeile weitergeschrieben.

\section*{Kapitel und Abschnitte}
Kapitel werden durch den Befehl {\color{blue}\verb|\chapter{}|} gebildet, dabei wird auch jedes Mal dem Namen eine Zahl zugeordnet, möchte man dies umgehen, schreibt man {\color{blue}\verb|\chapter*{}|}.\\
Das Gleiche gilt auch für die Abschnitte {\color{blue}\verb|\section{}|} und {\color{blue}\verb|\subsection{}|}, der Name ist dabei fett gedruckt und größer als der normale Text.



\section*{Kommentare}
Kommentare werden mit einem {\color{blue}\verb|%|} - Zeichen eingeleitet und dienen dem Schreiber.
\newline
\underline{Beispiel:} {\color{blue}\verb|%Dieser Befehl dient dem Einfügen von Bildern.|}

\section*{Graphiken}
Bilder o. Ä. können erst eingefügt werden, wenn ein Paket mit vordefinierten Befehlen am Anfang mit eingebunden wird.\\
\underline{Schreibe am Anfang:} {\color{blue}\verb|\usepackage{graphicx}|}
\\Um das Bild einfügen zu können, muss man angeben, wo die Graphik in deinen Dateien liegt.\\
\underline{Den Weg zum Bild angeben:} {\color{blue}\verb|\graphicspath{{Downloads/}}|}\\
Das Bild liegt hier in einem Ordner "Downloads", danach wird das Bild hochgeladen mit dem Angeben des Namens.\\ Zusätzlich kann die Größe des Bildes durch {\color{blue}"width="} angegeben werden, in diesem Fall ist das Bild 50\verb|%| so groß, wie der Text, auch kann man sich auf die Texthöhe mit {\color{blue}"height=\verb|\textheight"|} beziehen. \\
\underline{Einfügen des Bildes:} {\color{blue}\verb|\includegraphics[width=0.5\textwidth]{LaTeX-logo}|}\\
\underline{Beispiel:}
\graphicspath{{resources}}
\begin{figure}[h]
\includegraphics[width=0.5\textwidth]{LaTeX-logo}
\centering
\end{figure}\\
Mit {\color{blue}\verb|\begin{figure}[h]|} und {\color{blue}\verb|\end{figure}|} kann man noch den Ankerpunkt des Bildes angeben.\\
{\color{blue}t} = am Anfang der Seite  |  {\color{blue}b} = am Ende der Seite \\
{\color{blue}h} = ungefähr an der Stelle, an welcher man es im Code geschrieben hat\\
\vspace{0.5cm}
Das Bild kann auch benannt werden, einer Zahl zugewiesen werden und auf die Abbildung kann verwiesen werden.\\

\graphicspath{{resources/}}
\begin{figure}[h]
\centering
\includegraphics[width=0.75\textwidth]{LaTeX-logo}
\begin{center}Figure 1: Mit {\color{blue}\verb|\caption{}|} erstellt man diesen Text.\end{center}
\end{figure}
\raggedright
\label{fig:LaTeX-logo}
Mit {\color{blue}\verb|\label{fig:LaTeX-logo|} wurde dem Bild eine Nummer zugewiesen und mit {\color{blue}\verb|\ref{fig:LaTeX-logo|} wird diese Zahl hier ausgegeben: 1\ref{fig:LaTeX-logo}.\\

\section*{Listen}
Bei Listen muss zuerst eine Umgebung erstellt werden, also ein Bereich, in welchem die Liste sich befindet.
\begin{itemize}
    \item Zuerst schreibt man also: {\color{blue}\verb|\begin{itemize}|} oder {\color{blue}\verb|{enumerate}|}.
    \item Mit {\color{blue}\verb|\item|} werden dann die einzelnen Listenpunkte definiert.
    \item Um die Liste zu beenden, schreibt man dann: {\color{blue}\verb|\end{itemize}|} oder {\color{blue}\verb|{enumerate}|}.
\end{itemize}
\begin{enumerate}
    \item {\color{blue}itemize} kreiert Punkte und {\color{blue}enumerate} kreiert Nummern.
\end{enumerate}

\section*{Mathematik}
Um auch etwas Mathematik hier einfließen zu lassen, kann man Formeln darstellen und wenn man das im Satz tun möchte, passiert das am leichtesten durch das einklammern mit den {\color{blue}\verb|$$|}-Zeichen.\\ Möchte man eine mathematische Formel als einzigen Inhalt einer Zeile haben, nutzt man {\color{blue}\verb|\[...\]|} zum Einklammern.\\
\underline{Beispiel:}\\
Der Flächeninhalt eines Kreises wird durch die Formel $A=\pi×r^2$ beschrieben.\\
Der Flächeninhalt eines Kreises wird durch die Formel:
\[A=\pi×r^2\] beschrieben.

\section*{Tabellen}
Um eine Tabelle zu erstellen, braucht es wieder die Eingrenzung einer Umgebung, durch {\color{blue}\verb|\begin{tabular}{c c c}|} oder {\color{blue}\verb|{c r c}|} oder {\color{blue}\verb|{l r l}|}, usw.\\
Die zweite Klammer gibt an, ob der Inhalt einer Tabellenzelle {\color{blue}\underline{c}}entriert, auf der {\color{blue}\underline{l}}inken oder auf der {\color{blue}\underline{r}}echten Seite ist.\\
In diesem Fall gibt es drei×drei Tabellenzellen.\\
Dann werden die einzelnen Tabellen beschrieben durch:\\
\color{blue}
\begin{verbatim}
cell1 & cell2 & cell3\\
cell4 & cell5 & cell6\\
cell7 & cell8 & cell9\\
\end{verbatim}
\color{black}
Durch {\color{blue}\verb|\end{tabular}|} wird die Tabellenumgebung geschlossen.

\subsection*{Grenzen}
Um die Tabelle vertikal einzugrenzen kann man am Anfang: {\color{blue}\begin{verbatim}{c|c|c|}\end{verbatim}} notieren und um horizontal einzugrenzen, schreibt man am Anfang und Ende der Zellenbeschreibung {\color{blue}\verb|\hline|}, das kann man auch doppelt schreiben.\\
\underline{Beispiel:}\\
\vspace{0.5cm}
\begin{tabular}{||c|c|c||}
\hline\hline
cell1 & cell2 & cell3\\
\hline
cell4 & cell5 & cell6\\
\hline
cell7 & cell8 & cell9\\
\hline\hline
\end{tabular}
\color{blue}
\begin{verbatim}
\begin{tabular}{||c|c|c||}
\hline\hline
cell1 & cell2 & cell3\\
\hline
cell4 & cell5 & cell6\\
\hline
cell7 & cell8 & cell9\\
\hline\hline
\end{tabular}\\
\end{verbatim}
\color{black}

\subsubsection*{Zellenverschmelzung}
Zellenfelder kann man durch den Befehl: {\color{blue}\verb|\multicolumn{2}}{c}{cell 2}|} miteinandervereinen.\\
\underline{Beispiel:}\\
\vspace{0.5cm}
\begin{tabular}{c|c|c}
\hline
\multicolumn{3}{c}{cell1,2,3}\\
\hline
cell4 & cell5 & cell6\\
cell7 & cell8 & cell9\\
\end{tabular}\\
\vspace{0.5cm}

\section*{Interessante Befehle}
\subsection*{Abstände}
Man kann sowohl horizontale oder vertikale Abstände einfügen, das macht man mit den Befehlen: {\color{blue}\verb|\hspace{1cm}|} oder {\color{blue}\verb|\vspace{1cm}|}\\
Das kann dann so \hspace{3cm} aussehen oder so \vspace{1cm}\\ aussehen.

\subsection*{Übergreifende Schriftart}
Durch {\color{blue}\verb|\textit{}|} können ganze Sätze kursiv geschrieben werden. Für kursiv benutzt man {\color{blue}textit}, für fett {\color{blue}textbf} und bei underline gibt man ja generell den Bereich an.\\
\underline{Beispiel:}\\
\textit{Dieser ganze Satz ist kursiv!!}\\
\textbf{Dieser ganze Satz ist fett gedruckt!!}

\subsection*{Befehle als normalen Text}
Um diesen Text in \LaTeX{} erstellen und dabei die Befehle nennen zu können, muss man wissen, wie man Befehle notiert, ohne dass diese als Befehl ausgeführt werden.\\
Jenes tut man durch den Befehl:\\
\color{blue}
\begin{verbatim} \verb|underline{}| \end{verbatim}
\color{black}
Falls man ganze Bereiche von Befehlen als Text möchte, erstellt man eine Umgebung mit {\color{blue}\verb|\begin{verbatim}|} und beendet sie {\color{blue}\verb|\end{verbatim}|}.

\subsection*{Farbe}
In diesem Handout nutzte ich die farbliche Hervorhebung von Befehlen, davor muss man das Paket {\color{blue}xcolor} durch {\color{blue}\verb|\usepackage{xcolor}|} einfügen.\\
Durch {\color{blue}\verb|\color|} wird dann der ganze folgende Text in der Wahlfarbe eingefärbt.\\
Auch hier kann man alternativ eine Umgebung mit {\color{blue}\verb|\begin{red}|} und mit {\color{blue}\verb|\end{red}|} erstellen.\\
Einzelne Worte können mit {\color{blue}\verb|{\color{red}...}|} koloriert werden, um einen farblichen Kasten zu erstellen benutzt man den Befehl {\color{blue}\verb|\colorbox{red}{}|}.\\
\underline{Beispiele:}\\
\color{blue} Dieser ganze Satz ist blau.\\
\color{black}
In diesem Satz ist nur das Wort {\color{red} Apfel} rot.\\
Und hier ist das letzte Wort in einem farblichen \colorbox{orange}{Kasten}.\\
\vspace{1cm}
\huge Viel Spaß mit \LaTeX{!!!}
\end{document}
