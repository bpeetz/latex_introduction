%! TEX root = ../main.tex
% LTeX: language=de-DE


%\begin{frame}{Warum dann eigentlich?}
%	\begin{center}
%		"What you see is what you get" (WYSIWYG) \pause \\
%		vs. \\
%		"What you see is what you mean" (WYSIWYM)
%	\end{center}
%\end{frame}

\setbeamercovered{transparent}
\begin{frame}
	\begin{description}
		\item<1,2>[WYSIWYG]
		\begin{itemize}
			\item<2>{Formatierung und Inhalt sind nicht getrennt.}
		\end{itemize}
		\item<3-6>[WYSIWYM]
		\begin{itemize}
			\item<4>{Inhalt und Formatierung sind streng getrennt.}
			\item<5>{Inhalt bekommt aber Semantik, nicht die Formatierung angegeben}
			\item<6>{Genaue Formatierung wird erst am Ende durch Definitionen der Dokumentklasse erkenntlich}
		\end{itemize}
	\end{description}
\end{frame}
\setbeamercovered{invisible}

\begin{frame}{WYSIWYM}
	\setbeamercolor{itemize/enumerate body}{fg=gray}
	\begin{itemize}[<+-|alert@+>]
		\item Von fast jeder wissenschaftlichen Zeitschrift genutzt (Science, Nature, etc.)
		\item Leicht, sich auf den Inhalt zu konzentrieren
        \item Macht Restrukturierung des Dokuments sehr einfach
	\end{itemize}
\end{frame}
