%! TEX root = ../main.tex
% LTeX: language=de-DE


% # Wahrgenommene Vorteile von Word:

% 1. Benutzerfreundlichkeit:
% Word bietet eine intuitivere Benutzeroberfläche, die es einfacher macht,
% Dokumente zu erstellen und zu bearbeiten, besonders für Menschen,
% die nicht mit Markup-Sprachen vertraut sind.

% 1. Breite Akzeptanz:
% Microsoft Word ist ein weit verbreitetes Textverarbeitungsprogramm,
% das von den meisten Menschen problemlos genutzt werden kann. Es ist die
% Standardlösung in vielen Unternehmen und Bildungseinrichtungen.

% 1. Einfache Formatierung:
% In Word kann man Texte einfach formatieren,
% indem man auf die Schaltflächen in der Menüleiste klickt, ohne
% sich mit speziellen Befehlen oder Codes befassen zu müssen.

% 1. Kompatibilität:
% Word-Dokumente können leicht in verschiedene
% Dateiformate (wie PDF oder RTF) exportiert werden und sind somit
% mit einer Vielzahl von Geräten und Plattformen kompatibel.

% 1. Einfache Zusammenarbeit:
% Word bietet Funktionen zur gemeinsamen
% Bearbeitung und Kommentierung von Dokumenten, die die Zusammenarbeit in
% Echtzeit ermöglichen.

% 1. Visuelles Feedback:
% Word zeigt sofort das endgültige Layout des Dokuments,
% während man daran arbeitet, was für einige Benutzer:innen
% hilfreich sein kann.

% # Wahrgenommene Nachteile von LaTeX:

% 1. Einstiegshürde:
% Der Einstieg in LaTeX kann für Anfänger:innen anfangs schwierig sein,
% da es eine Markup-Sprache ist und nicht so intuitiv wie herkömmliche
% Textverarbeitungsprogramme funktioniert.

% 1. Komplexität:
% LaTeX kann bei umfangreichen oder komplexen Dokumenten eine steile Lernkurve haben,
% insbesondere wenn spezielle Anpassungen oder individuelle Formatierungen
% erforderlich sind.

% 1. Visuelles Feedback:
% Im Gegensatz zu
% WYSIWYG-Editoren wie Word zeigt LaTeX möglicherweise nicht sofort
% das endgültige Layout des Dokuments, was für einige Nutzer:innen
% unpraktisch sein kann.

% 1. Einschränkungen bei Zusammenarbeit:
% Wenn Teammitglieder oder Koautor:innen nicht mit LaTeX vertraut sind, kann
% die Zusammenarbeit schwierig sein, da die Bearbeitung und Kommentierung
% möglicherweise nicht so nahtlos ist wie in Word.

% 1. Begrenzte Vorlagen:
% Standardmäßig bietet LaTeX weniger vorgefertigte Vorlagen und
% Layouts im Vergleich zu Word oder anderen Textverarbeitungsprogrammen,
% was die Erstellung bestimmter Dokumenttypen etwas zeitaufwendiger machen
% kann.

% 1. Fehlende Sofort-Korrektur:
% Während WYSIWYG-Editoren oft
% Rechtschreib- und Grammatikprüfung in Echtzeit bieten, ist dies bei
% LaTeX nicht der Fall, es sei denn, man verwendet zusätzliche Tools.

% 1. Grafiken und Tabellen:
% Das Einbinden von Grafiken und Tabellen
% kann in LaTeX manchmal komplizierter sein als in WYSIWYG-Editoren,
% insbesondere wenn es um die Feinabstimmung der Platzierung geht.

% 1. Abhängigkeit von Markup:
% Die Notwendigkeit, spezielle Markup-Befehle
% zu verwenden, kann für einige Nutzer:innen mühsam sein, besonders wenn
% sie nicht oft mit LaTeX arbeiten.

% 1. Zeit- und Arbeitsaufwand:
% Die Erstellung komplexer Layouts kann in LaTeX mehr Zeit und Aufwand
% erfordern als in WYSIWYG-Editoren, wo visuelle Änderungen oft schneller
% durchgeführt werden können.

% 1. Fehler:
% Wenn die LaTeX-Umgebung oder die verwendeten Pakete nicht richtig eingerichtet
% sind, können unerwartete Fehler auftreten, die überaus schwer zu debuggen sind.

\begin{frame}{Wahrgenommene Nachteile}
	\setbeamercolor{itemize/enumerate body}{fg=gray}
	\begin{enumerate}[<+-|alert@+>]
		\item "Nur" eine Programmiersprache:
		      \begin{itemize}
                    \item Kein Point-und-Klick Interface.
                    \item Kein einheitlicher Text-Editor.
                    \item Unterschiedliche Distributionen (TeX Live, MikTeX, etc.).
                    \item Verschieden Compiler (pdfLaTeX, LuaLaTex, XeLaTeX, etc.)
		      \end{itemize}
        \item Syntax-Fehler
	\end{enumerate}
\end{frame}
