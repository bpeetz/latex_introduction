%! TEX root = ../main.tex
% LTeX: language=de-DE


\begin{frame}{Warum dann eigentlich?}
	\begin{center}
		"What you see is what you get" (WYSIWYG) \pause \\
		vs. \\
		"What you see is what you mean" (WYSIWYM)
	\end{center}
\end{frame}

\setbeamercovered{transparent}
\begin{frame}
	\begin{description}
		\item<1,2>[WYSIWYG]
		\begin{itemize}
			\item<2>{Formatierung und Inhalt sind nicht getrennt.}
		\end{itemize}
		\item<3-6>[WYSIWYM]
		\begin{itemize}
			\item<4>{Inhalt und Formatierung sind streng getrennt.}
			\item<5>{Formatierung ergibt sich durch semantische Befehle.}
			\item<6>{Genaue Formatierung wird erst am Ende durch Definitionen der Semantik erkenntlich}
		\end{itemize}
	\end{description}
\end{frame}
\setbeamercovered{invisible}
